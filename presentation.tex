\documentclass[12pt]{beamer}

\usepackage[english]{babel}
\usepackage[utf8]{inputenc}
\usepackage[T1]{fontenc}
\usepackage{lmodern}
\usepackage[export]{adjustbox}

\usetheme{Berlin}
\usecolortheme{spruce}
\useoutertheme{infolines}

\title{how do we know how far things are}
\author{Ethan Snyder}
\date{\today}

\begin{document}
%\setbeamertemplate{headline}{}

\begin{frame}
    \titlepage
\end{frame}

\section{Introduction}

    \begin{frame}
        \centering
        \large how do we know how far things are?\\
        \vspace{2em}
        \normalsize this will be our leading question\\
        \vspace{2em}
        enjoy :)
    \end{frame}
    \begin{frame}
        \centering
        parallax, mostly
    \end{frame}
    \subsection{Parallax --- Moving \& Stationary}
        \begin{frame}{parallax} \centering
            this may be what you're picturing :)
            \begin{figure}
                \includegraphics[scale=0.8, fbox, bb=0 0 240 160]{parallaxcar.jpeg}
                \caption{Parallax as seen looking out a moving car's window$^0$.}
            \end{figure}
            \footnotetext{\tiny Image: https://stock.adobe.com/images/view-out-the-car-window-as-the-scenery-blurs-by/193746850}
        \end{frame}
        \begin{frame}{parallax} \centering
            two types of parallax
            \pause
            \begin{columns}
                \begin{column}{0.5\linewidth}
                    \begin{block}{moving parallax}
                        \begin{itemize}
                            \item<3-> involves movement
                            \item<4-> things close to observer appear to move more, things farther appear to move less
                        \end{itemize}
                    \end{block}
                \end{column}
                \begin{column}{0.5\linewidth}
                    \begin{block}{stationary parallax}
                        \begin{itemize}
                            \item<3-> does not
                            \item<4-> the change in an objects appearance from two different locations (at once or at different times)
                        \end{itemize}
                    \end{block}
                \end{column}
            \end{columns}
            \pause[5] \vspace{1em}
            (these are actually the same, kinda. motion is just being in two places at different times :) )
        \end{frame}
        \begin{frame}{moving parallax} \centering
            moving parallax car picture again here look
            \begin{figure}
                \includegraphics[scale=0.8, fbox, bb=0 0 240 160]{parallaxcar.jpeg}
                \caption{Parallax as seen looking out a moving car's window again$^0$.}
            \end{figure}
            \footnotetext{\tiny Image: https://stock.adobe.com/images/view-out-the-car-window-as-the-scenery-blurs-by/193746850}
        \end{frame}
        \begin{frame}{stationary parallax} \centering
            like human eyes, for example\\
            this is how depth perception works
            \begin{figure}
                \includegraphics[scale=0.06, bb=0 0 1400 2000, fbox]{parallaxeyes.jpg}
                \caption{Eyes doing parallax$^0$.}
            \end{figure}
            \footnotetext{\tiny Image: https://cdn.britannica.com/85/4085-050-5575ACA3/parallax-points-NL-eyes-NR-left.jpg}
        \end{frame}
        \begin{frame}{summary} \centering
            we know how far things are away from us\\because we have EYES dipshit
        \end{frame}
        \begin{frame} \centering
            ok but what about numbers\\
            \vspace{2em}
            like what if we want to MEASURE a distance
        \end{frame}
        \begin{frame}
            \tableofcontents
        \end{frame}
\section{Measuring Shortish Distances}
    \subsection{Apparent Size, Units, Measuring Devices}
        \begin{frame}{Apparent Size}{What if you \textit{know the size of a distant object?}} \centering
            Easiest solution for measuring a distance.
            \begin{figure}
                \includegraphics[scale=0.6, bb=0 0 280 160]{angletriangle.png}
            \end{figure}
            \pause
            \[\sin{\theta}=\frac{h}{d} \implies d=\frac{h}{\sin{\theta}}\]
            \pause
            (you just need to know the size of the distant object and be able to measure the \textit{angular extent} $\theta$)
        \end{frame}
        \begin{frame}{dawg wt f}{what if I don't have a protractor or konw the size of the thing?}
            \begin{figure}
                \includegraphics[scale=0.2, bb=0 0 800 600]{confused.png}
                \caption{Man scratching his head, confused about how to measure a distance without prior knowledge of a distant object's size or the ability to measure angular extent$^0$.}
            \end{figure}
            \footnotetext{\tiny Image: https://stock.adobe.com/images/portrait-of-a-mixed-race-man-scratching-his-head-in-confusion/68695988}
        \end{frame}
    \subsection{Units}
        \begin{frame}
            \begin{figure}
                \includegraphics[scale=0.1]{tapemeasure.jpg}
            \end{figure}
        \end{frame}
        \begin{frame}
            \begin{figure}
                \includegraphics[scale=0.5]{rollermeasure.png}
            \end{figure}
        \end{frame}
        \begin{frame}{Units :)}{How many \textit{things} span this distance?}
            \begin{itemize}
                \item<2-> Find something to use as a unit
                \item<3-> Find out how many fit span a distance
                \begin{itemize}
                    \item Put many of these objects between and count the number
                    \item Use the same object and mark intervals of the object, count the intervals
                \end{itemize}
                \item<4-> Historically we've used things like a foot but truly make up anything
                \item<5-> If you want other people to use this unit just make sure their unit is the same as yours
            \end{itemize}
        \end{frame}
    \subsection{Measuring Devices}
        \begin{frame}{Measuring Devices}
            \begin{figure}
                \includegraphics[scale=0.1]{tapemeasure.jpg}
                \includegraphics[scale=0.5]{rollermeasure.png}
                \caption{A tape measure and a rolly measurey thing$^1$.}
            \end{figure}
            \footnotemark{\tiny Image: https://cdn.shopify.com/s/files/1/0404/0048/6557/products/7\_CDM1001\_1024x1024\@2x.png?v=1615341175}
        \end{frame}
        \begin{frame}{how far is this fucker?}
            \begin{figure}
                \includegraphics[scale=0.45, fbox]{moon.jpg}
                \caption{Moon$^2$.}
            \end{figure}
            \footnotemark{\tiny Image: Me :)}
        \end{frame}
\section{The Distance Ladder}
    \subsection{Background}
        \begin{frame}{The Distance Ladder}{Earth $\implies$ Moon $\implies$ Sun $\implies$ Planets $\implies$ Stars $\implies$ Galaxies} \centering
            \pause
            We need to know each distance/size in order to find out the next object's size/distance.\\
            \pause
            The goal of measuring the Earth, Moon, Sun, and stars is as old as humanity.\\ \vspace{2em}
            \small (so let's go back)
        \end{frame}
        \begin{frame}{What have we always known?} \centering
            (This is essentially the same question as what can be observed with the eyes.)
            \pause
            \begin{itemize}
                \item<2-> The Moon is closer than the Sun
            \end{itemize}
        \end{frame}
        \begin{frame} \centering
            Solar eclipses, the moon passing between the Earth and Sun
            \begin{figure}
                \includegraphics[scale=0.17, fbox]{eclipse.jpg}
            \caption{Partial, total, and annular eclipse images$^3$.}
            \end{figure}
            \footnotemark{\tiny Image: https://galamcdougal.blogspot.com/2022/10/solar-eclipse.html}
        \end{frame}
        \begin{frame}{What have we always known?} \centering
            (This is essentially the same question as what can be observed with the eyes.)
            \begin{itemize}
                \item The Moon is closer than the Sun (solar eclipses) and \textit{same angular size}
                \item The Earth is round
            \end{itemize}
        \end{frame}
        \begin{frame} \centering
            During a lunar eclipse, the Earth passes between a full Moon and the Sun
            \begin{figure}
                \includegraphics[scale=0.15, fbox]{lunar2.JPG}
                \includegraphics[scale=0.15, fbox]{lunar1.JPG}
                \caption{Two stages of a lunar eclipse$^4$.}
            \end{figure}
            \footnotemark{\tiny Images: me :)}
        \end{frame}
        \begin{frame}{What have we always known?} \centering
            (This is essentially the same question as what can be observed with the eyes.)
            \begin{itemize}
                \item The Moon is closer than the Sun (solar eclipses) and \textit{same angular size}
                \item The Earth is round (lunar eclipses)
                \item The Sun, Moon, planets, and stars are cyclical and move in `perfect circles' around the Earth on a flat plane
                \pause
                \begin{itemize}
                    \item we were wrong about this but it was a very compelling, mostly unproblematic explanation for a LONG time
                    \pause
                    \item we also knew slower moving things were farther (parallax), so we knew Jupiter and Saturn are far and Venus and Mars are close
                \end{itemize}
            \end{itemize}
        \end{frame}
        \begin{frame} \centering
            \begin{figure}
                \includegraphics[scale=0.25, fbox]{geocentric.png}
                \caption{Geocentric model with subcircles that explain retrograde motion$^5$.}
            \end{figure}
            \footnotemark{\tiny Image: https://starrythoughts.weebly.com/uploads/1/6/3/0/16304784/2180964\_orig.gif}
        \end{frame}
        \begin{frame} \centering
            \begin{figure}
                \includegraphics[scale=0.3, fbox]{ecliptic.png}\\
                \includegraphics[scale=0.08, fbox]{eclipticsky.jpg}
                \caption{Ecliptic visuals$^6$.}
            \end{figure}
            \footnotemark{\tiny Images: https://www.astronomynotes.com/nakedeye/phases/solarsys.gif, NASA}
        \end{frame}
        \begin{frame}{In summary,}{(this information is all we needed to figure out the distance to the stars. the rest is all measurements and math)}
            \begin{itemize}
                \item The Earth is spherical
                \item The Moon is closer than the Sun and \textit{same angular size}
                \item All solar system bodies orbit in a flat plane in perfect circles around the Earth
                \pause
                \begin{itemize}
                    \item (no physics behind this, but it was a pretty viable, elegant explanation)
                \end{itemize}
            \end{itemize}
        \end{frame}
    \subsection{Earth}
        \begin{frame}{The First Recorded Attempt of Measuring Things}{Aristarchus of Samos, Greek, 270BC, Heliocentrist}
            Realized the distance of the sun would change when we observe a half-moon. Instead of starting humble with the size of the Earth bro went wacko and tried to get THIS distance first.
            \begin{figure}
                \includegraphics[scale=0.19, frame]{nearsun.png}
                \includegraphics[scale=0.19, frame]{farsun.png}
                \caption{Near-Sun rays make a half-moon appear sooner in the lunar cycle, while far-Sun parallel rays mean half-moons happen at $\frac{1}{2}$ and $\frac{3}{4}$ in the lunar cycle.}
            \end{figure}
        \end{frame}
        \begin{frame}{Aristarchus' Method} \centering
            Measure the angle $\theta$ at exactly half-moon.
            \begin{figure}
                \includegraphics[scale=0.3, frame]{aristocrates.png}
            \end{figure}
            \[\sin{\theta}=\frac{l}{d} \implies \frac{l}{d}=\sin{\theta}\]
        \end{frame}
        \begin{frame}{The First Recorded Attempt of Measuring Things}{Aristarchus of Samos, Greek, 270BC, Heliocentrist}
            The issue here is incredibly precise timing --- when is it exactly half-moon?\\ \vspace{1em}
            \pause
            The lunar cycle is 29.5 days (which was known)\\
            Nowadays, we know that half-moons occur 30 mins after it reaches those 90$^\circ$ marks\\
            This meant measuring $\frac{1}{1400}$th of a lunar cycle.\\
            \vspace{1em}
            \pause
            Aristarchus measured $87^\circ$, meaning $\frac{l}{d}=\sin{(90^\circ-87^\circ)}=19.1$ the Sun is 19x farther away than the Moon.\\
            In reality, this half-moon angle is 89.88$^\circ$, which gives a much larger number: the Sun is 409x farther away than the Moon.
        \end{frame}
        \begin{frame}{The First Recorded Attempt of Measuring Things}{Aristarchus of Samos, Greek, 270BC, Heliocentrist} \centering
            good effort dipshit
        \end{frame}
        \begin{frame}{Eratosthenes and the Size of the Earth}{Greece, 230BC} \centering
            The first good measurement of the size of the Earth
            \pause
            \begin{figure}
                \includegraphics[scale=0.35, frame]{eritosthenesdiagram.png}
                \caption{Eratosthenes' experimental setup$^7$.}
            \end{figure}
            \footnotemark{\tiny Image: https://www.mezzacotta.net/100proofs/images/002-SyeneAlexandria.png}
        \end{frame}
        \begin{frame}{Eratosthenes and the Size of the Earth}{Greece, 230BC} \centering
            Eratosthenes knew the city of Syene was on the Tropic of Cancer (no shadows on summer solstice, sun right overhead)\\ \vspace{1em}
            
            He also knew the distance between Alexandria and Syene\\

            \begin{figure}
                \includegraphics[scale=0.1, frame]{egyptmap.png}
                \caption{Alexandria and Syene along the Nile River in Egypt$^8$.}
            \end{figure}
            \footnotemark{\tiny Image: https://www.researchgate.net/profile/Alok-Kumar-111/publication/253596660/figure/fig2/}
        \end{frame}
        \begin{frame}{Eratosthenes' Method}
            \begin{columns}
                \begin{column}{0.5\linewidth}
                    \begin{figure}
                        \includegraphics[scale=0.4, frame]{earthdiam.png}
                    \end{figure}
                \end{column}
                \begin{column}{0.5\linewidth}
                    Arc length:\\
                    \[s=\frac{{\pi}R\theta}{180^\circ}\]
                    Rearrange for R, Earth's radius:\\
                    \[R=\frac{180s}{\pi\theta}\]
                    $s$ was already known from geographical data, and $\theta$ was measureable from shadows cast in Alexandria.
                \end{column}
            \end{columns}
        \end{frame}
        \begin{frame}{Eratosthenes and the Size of the Earth}{Greece, 230BC} \centering
            To avoid doing difficult division with $\pi$, he opted to calculate the circumference, not the radius:
            \[C=2{\pi}R=\frac{360s}{\theta}\]
            To which he found a value of 252,000 stadia.\\
            In ancient Greece, they used a unit called a `stade'. We don't really know how big a stade was, but we estimate 1 stade is about 525ft, making his calculation \textit{less than 1\% off.}\\ \vspace{1em}
            Eratosthenes: 25,050mi.\\
            Modern value: 24,901mi.
        \end{frame}
    \subsection{Moon}
        \begin{frame}{Hipparchus and the Distance to the Moon}{Greece, 189BC} \centering
            Looking at a solar eclipse from two different places allowed for the calculation of the distance to the moon
        \end{frame}
    \subsection{Sun}
    \subsection{Planets}
    \subsection{Stars}
    \subsection{Galaxies}

\end{document}